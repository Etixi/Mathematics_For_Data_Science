\documentclass[11pt]{article}
\usepackage{amsmath}  % Ajout du package amsmath pour \dfrac
\pagestyle{empty}


\begin{document}

Exposants 
$$2x^3$$ 
$$2x^{34}$$ 
$$2x^{3x+4}$$ 
$$2x^{3x^4+5}$$ 

Indices 
$$x_1$$ 
$$x_{12}$$ 
$$x_{1_2}$$ 
$$x_{1_{2_3}}$$
$$a_0, a_1, a_2, \ldots, a_{100}$$

Lettres grecques
$$\pi$$
$$\Pi$$
$$\alpha$$
$$A=\pi r^2$$

Fonctions trigonométriques
$$y=\sin x$$
$$y=\cos x$$
$$y=\csc\theta$$
$$y=\sin^{-1} x$$
$$y=\arcsin x$$

Fonctions logarithmiques
$$y=\log x$$
$$y=\log_5 x$$
$$y=\ln x$$

Racines
$$\sqrt{2}$$
$$\sqrt[3]{2}$$
$$\sqrt{x^2+y^2}$$
$$\sqrt{ 1+\sqrt{x} }$$. \\[16pt]

Fractions
$$\frac{2}{3}$$
Environ $\displaystyle \frac{2}{3}$ du verre est plein.\\[16pt]
Environ $\frac{2}{3}$ du verre est plein.\\[6pt]
Environ $\dfrac{2}{3}$ du verre est plein.

$$\frac{\sqrt{x+1}}{\sqrt{x+2}}$$
$$\frac{\sqrt{x+1}}{\sqrt{x+2}}$$
$$\frac{1}{ (1+\frac{1}{x}) }$$




\end{document}