\documentclass[11pt]{article}
\usepackage{hyperref} % Ajout du package pour \href

\title{Mon Document {\LaTeX}}
\author{Etienne KOA}
\date{\today}

\begin{document}
\tableofcontents
\maketitle

Cela produira du texte en \textit{italique}.

Cela produire du texte en \textbf{gras}.

Cela produira du texte en \textsc{minuscule}.

Cela produira du texte en \texttt{police}.

Veuillez visiter le site Web de Michelle Krummel à \href{http://michellekrummel.com}{site Web}.

\vspace{1cm}

Veuillez excuser ma chère tante Sally.

Veuillez excuser ma \begin{Large} chère tante Sally \end{Large}.

Veuillez excuser ma \begin{Huge} chère tante Sally \end{Huge}.

Veuillez excuser ma \begin{Huge} chère tante Sally \end{Huge}.

Veuillez excuser ma \begin{normalsize} chère tante Sally \end{normalsize}.

Veuillez excuser ma \begin{small} chère tante Sally \end{small}.

Veuillez excuser ma \begin{scriptsize}chère tante Sally \end{scriptsize}.

Veuillez excuser ma \begin{tiny} chère tante Sally \end{tiny}.

\vspace{1cm}

\begin{center} Cette ligne est centré.\end{center}
\begin{flushleft} Cette ligne est justifié à gauche.\end{flushleft}
\begin{flushright} Cette ligne est justifié à droite.\end{flushright}


\vspace{1cm}

\Large
Cette ligne est centré.\\
Cette ligne est justifié à gauche.\\
Cette ligne est justifié à droite.

\vspace{1cm}

\tiny
Cette ligne est centré.\\
Cette ligne est justifié à gauche.\\
Cette ligne est justifié à droite.

\vspace{1cm}



\section{Fonction Linéaire}

	\subsection{Forme d'interception de pente}
	
		\subsubsection{Exemple 1:}
		\subsubsection{Exemple 2:}
	\subsection{Forme standard}
	
	\subsection{Forme point-pente}
	
\section*{Fonctions Quadratiques}

	\subsection{Forme Vertex}
	\subsection{Forme Standard}
	\subsection{Forme Factorisée}

\end{document}